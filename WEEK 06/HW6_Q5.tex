\documentclass[12pt, letterpaper, twoside]{article}
\usepackage[utf8]{inputenc}
\usepackage[margin=1in]{geometry}
\usepackage{enumitem}
\usepackage{siunitx}
\usepackage{xcolor}
\usepackage{amsmath}
\usepackage{amssymb}
\usepackage{mathtools}
\usepackage{blkarray, bigstrut}
\usepackage{array}
\usepackage{lmodern}
\usepackage{array,mathtools}
\usepackage{mathtools}
\DeclarePairedDelimiter{\ceil}{\lceil}{\rceil}

\newcommand*{\carry}[1][1]{\overset{#1}}
\newcolumntype{B}[1]{r*{#1}{@{\,}r}}
\sisetup{
  exponent-product=\cdot,
  tight-spacing,
  per-mode=symbol,
}


\title{\textbf{Homework 5}}
\author{Oscar Ramirez}
\date{July 22 2022}

\begin{document}
\begin{titlepage}   

    \raggedleft % Right align the title page

    \rule{1pt}{\textheight} % Vertical line
    \hspace{0.05\textwidth} % Whitespace between the vertical line and title page text
    \parbox[b]{0.75\textwidth}{ % Paragraph box for holding the title page text, adjust the width to move the title page left or right on the page
        {\large\textit{}}\\[4\baselineskip]

        {\Huge\bfseries Homework 6}\\[2\baselineskip] % Title

        {\Large\textsc{Oscar Ramirez}} % Author name, lower case for consistent small caps

        \vspace{0.47\textheight} % Whitespace between the title block and the publisher
        {\noindent Question 5}\\[0.1\baselineskip]
        {\noindent NYU Tandon CS Extended Bridge Summer 2022}\\[0.1\baselineskip]
        {\noindent NetID: or2092}\\[\baselineskip]
    }

\end{titlepage}
\newpage

\noindent \textbf{\underline{Question 5:}}\\
Use the definition of $\theta$ in order to show the following:\\
a. $5n^3 + 2n^2 + 3n$ = $\theta(n^3)$\\
\break
Proof:\\
\centerline{Let $f(n) = 5n^3 + 2n^2 + 3n$ and $g(n) = (n^3)$, }\\
\centerline{we will prove that $c_{2} * g(n) \leq f(n) \leq c_{1} * g(n)$ for any $n_{0} \geq 1$. }\\
\break
\centerline{For any $n \geq 1$, we know that $5n^3 \leq 5n^3 + 2n^2 + 3n$.}\\
\centerline{So if we take $c_{2} = 5$, and $n_{0} = 1$, then $5n^3 \leq 5n^3 + 2n^2 + 3n$.}\\ 
\centerline{\textbf{Therefore, } $\mathbf{5 * g(n) \leq f(n)}$ \textbf{ and } $\mathbf{f = \Omega(g)}$}\\
\break
\centerline{For any $n \geq 1$, we know that $n^2$ and $n^3$ are larger than n.}\\
\centerline{So, we know that $5n^3 + 2n^2 + 3n \leq 5n^3 + 2n^3 + 3n^3$.}\\
\centerline{So if we take $c_{1} = 10$ and $n_{0} = 1$, then $5n^3 + 2n^2 + 3n \leq 10n^3$.}\\
\centerline{\textbf{Therefore,} $\mathbf{f(n) \leq 10 * g(n)}$ \textbf{and} $\mathbf{f = O(g)}$}\\
\break
\centerline{There for if we take $c_{1} = 10$, $c_{2} = 5$ and $n_{0} = 1$, }\\
\centerline{then for all $n \geq n_{0}$, $f = O(g)$ and $f = \Omega (g)$,}\\
\centerline{\textbf{Therefore,} $\mathbf{f = \theta(g) = \theta(n^3)}$}\\
\newpage
\noindent\textbf{b.} $\sqrt{7n^2 + 2n - 8}$ = $\theta(n)$\\
\break
\break
Proof:\\
\centerline{Let $f(n) = \sqrt{7n^2 + 2n - 8}$ and $g(n) = (n)$, }\\
\centerline{we will prove that $c_{2} * g(n) \leq f(n) \leq c_{1} * g(n)$ for any $n_{0} \geq ?$. }\\
\break
\centerline{Using the expression $7n^2 + 2n - 8$, we can see that $7n^2 \leq 7n^2 + 2n - 8$,}\\
\centerline{when $2n - 8 \geq 0$. This gives us $n_{0} = \sqrt{4} = 2$. So we take $c_{2} = \sqrt{7n^2}$}.\\
\centerline{and round $\sqrt{7}$ down to 2 and take $c_{2} = 2$. So, for all $n \geq 2$, $2n \leq  \sqrt{7n^2 + 2n - 8}$.}
\centerline{\textbf{Therefore, } $\mathbf{2 * g(n) \leq f}$ \textbf{ and } $\mathbf{f = \Omega(g)}$}\\
\break
\centerline{Using the expression $7n^2 + 2n - 8$, we know that $7n^2 + 2n - 8 \leq 7n^2 + 2n^2$.}\\
\centerline{So, we know that $7n^2 + 2n - 8 \leq 9n^2$. $\sqrt{9n^2}$ gives us $3n$.}\\
\centerline{So if we take $c_{1} = 3$ and $n_{0} = 2$, then $\sqrt{7n^2 + 2n - 8} \leq 3n$.}\\
\centerline{\textbf{Therefore,} $\mathbf{f(n) \leq 3 * g(n)}$ \textbf{and} $\mathbf{f = O(g)}$}\\
\break
\centerline{There for if we take $c_{1} = 3$, $c_{2} = 2$ and $n_{0} = 2$, }\\
\centerline{then for all $n \geq n_{0}$, $f = O(g)$ and $f = \Omega (g)$,}\\
\centerline{\textbf{Therefore,} $\mathbf{2 * g(n) \leq f \leq 3 * g(n)}$ and $\mathbf{f = \theta(g) = \theta(n)}$}\\
\end{document}